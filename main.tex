\documentclass[10pt, a4paper]{article}
\usepackage{lecture}

\title{Lecture Note: Introductory of Functional Analysis}
\author{Terence, Junjie LIU$^1$\thanks{Author is a Mphil student in Probability Theory and Mathematical Statistics. Considered of my limited ability, mistakes are inevitable. Correction and suggestions are welcomed. This work is under the CC-BY-SA 4.0 International license \faCreativeCommons\ \faCreativeCommonsBy\ \faCreativeCommonsSa.} \\ \href{terencelau@uicstat.com}{terencelau@uicstat.com}}
\date{$^1$ Department of Statistics, BNU-HKBU United International College, Zhuhai, China \\ Date of Final Compile: \today}

\begin{document}

\maketitle 

\begin{abstract}
\noindent Functional analysis is a rich subject because it combines two large branches of mathematic: the \textit{topological} branch concerns itself with convergence, continuity,
connectivity, boundedness, etc.; and the \textit{algebraic} branch concerns itself with operations, groups, rings, vectors, etc. Since this is the introductory course on Functional Analysis, in this course, we mainly focus on these parts:

    \begin{enumerate}
        \item Some Preliminaries in Analysis
        \item Metric space
        \item Norm Space and Banach Space
        \item Linear vector space, Linear Operator and Dual Space
        \item Inner Product space and Hilbert Space
    \end{enumerate}
    
\noindent Here are some reference books and lecture notes:
    
    \begin{enumerate}
        \item Books\footnote{The highlighted implies ``highly recommended"}:
        \begin{enumerate}
            \item \textit{Tong Sun - Functional Analysis \cite{sun2010functional}}
            \item \textit{Yisheng Huang - Functional Analysis An Introduction Second Edition \cite{huang2009functional}}
            \item \textit{E. Kreyszig - Introductory Functional Analysis with Application \cite{kreyszig1978introductory}}
            \item \textit{Joseph Muscat - \hl{Functional Analysis - An Introduction to metric Spaces, Hilbert Spaces, and Banach Algebras}\cite{muscat2014functional}}
            \item \textit{Philippe G. Ciarlet - \hl{Linear and Nonlinear Functional Analysis with Applications}\cite{ciarlet2013linear}}
            \item \textit{Terence Tao - \hl{Analysis}\cite{tao2009analysis}}
            \item \textit{Matthew A. Pons - \hl{Real Analysis for the Undergraduate} \cite{pons2014real}}
        \end{enumerate}
        \item Notes:
        \begin{enumerate}
            \item \href{http://www.math.utah.edu/~tan/notes.html}{Mr. Chee Han Tan's note (PhD Candidate in Utah University)}
            \item \href{http://math.wsu.edu/students/jstreipel/notes/functionalanalysis.pdf}{Mr. Jakob Streipel's Note (Doctoral student in Washington State University)} 
            \item Dr.Man Wa HUI's Lecture note (BNU-HKBU UIC Course MATH7110 - Functional Analysis)
            \item \href{https://www.math.cuhk.edu.hk/course_builder/2122/math4010/Functional\%20analysis\%202021-22a.pdf}{Prof. Chi Wai LEUNG's Lecture note} (The Chinese University of Hong Kong Course MATH4010 - Functional Analysis)
        \end{enumerate}
        %\item Videos:
        %\begin{enumerate}
        %    \item \href{https://www.youtube.com/playlist?list=PLBh2i93oe2qsGKD%OsuVVw-OCAfprrnGfr}{\textit{Functional Analysis} by The Right Side of %Mathematics}
        %\end{enumerate}
    \end{enumerate}
    
However, most of the Statistics undergraduate in BNU-HKBU United International College did not take the course of real analysis, only basic Multivariate Calculus, thus, there are multiple preliminary's knowledge missing. In this note, I am trying to fill the gap by covering the missing chapters and I highly recommend those students who enrol this course spend some time on Mathematical Analysis and Real Analysis, also, take some time to do the revision on Linear Algebra.

As a guide, the notes and exercises have been marked as follows:

\begin{enumerate}
    \item[$\blacktriangleright$] refers to important notes and results, also \uwave{wave}, \uline{underline} or \hl{hightlight}.
    \item[$*$] more advanced or difficult exercises that can be skipped on a first reading;
    \item[$\diamondsuit$] side remarks that can be skipped without losing any essential ideas
\end{enumerate}

\noindent``数学当中最麻烦的事情就是显然,我觉得显然,你觉得不显然这就是最麻烦的事情了”\footnote{from article \href{https://zhuanlan.zhihu.com/p/38029151}{``选择公理与Zorn引理" by Diet Meat, posted in Zhihu}}

\end{abstract}

\newpage 

\begin{multicols}{2}
\tableofcontents 
\listoftheorems[ignoreall,show=definition]
\end{multicols}

\newpage

% part 1
\part{Preliminaries}
\section{Set Theory and Logic}

\subsection{Fundamental Concepts}

Here we introduce the ideas of set theory, and establish the basic terminology and notation. We also discuss some points of elementary logic.

\subsubsection{Basic Notation}

Commonly we shall use capital letters $A, B, \dots$ to denote \textbf{\uwave{set}}, and lowercase letters $a, b, \dots,$ to denote the \textbf{\uwave{elements}} belonging to these sets. If an element $a$ belongs to a set $A$, we  express this fact by the notation $a \in A$; for the contradiction, if $a$ does not belong to $A$, we express this fact by writing: $a \not\in A$. 

The equality symbol $=$ is used throughout this note to mean \uwave{logical identity}. Thus, when we say $a = b$, we mean $a$ and $b$ are symbols for the same element. Similarly, if $a$ and $b$ are different elements, we write $a \neq b$.

We say that $A$ is a \textbf{\uwave{subset}} if $B$ if every element of $A$ is also an element of $B$, denoted by $A \subset B$; ($\forall x \in A, x \in B$) Noting in this definition requires $A$ being different from $B$; in fact, \uline{if $A = B$, it is true that both $A \subset B$ and $B \subset A$.} If $A \subset B$ and $A$ is different from $B$, then, we say that $A$ is \textbf{\uwave{proper subset}} of $B$, denoted by $A \subseteq B$.\footnote{The relations $\subset$ and $\subseteq$ are called \textbf{\uwave{inclusion} and \uwave{proper inclusion}}, respectively.}

\subsubsection{Operations with Sets}

\paragraph{The Union of Sets and the meaning of "or"} Given two sets $A$ and $B$, one can form a set from them that consists of all the elements of $A$ together with all elements of $B$. This set is called \textbf{\uwave{the union}} of $A$ and $B$, is denoted by $A \cup B$. Formally, we define $A \cup B = \{x: x\in A, x \in B\}$. The union of two sets, means the elements of the new sets is from $A$, or from $B$, or from both $A$ and $B$.

\paragraph{The Intersection of Sets, and the Empty Set} Given sets $A$ and $B$, another way one can form a set is to take the common part of $A$ and $B$. This set is called the \textbf{\uwave{intersection}} of $A$ and $B$ and is denoted by $A \cap B$. Formally, we define $A \cap B = \{x: x\in A \text{ and } x \in B\}$. If the set $A$ and $B$ has no common element, then we call the intersection of $A$ and $B$ is a \textbf{\uwave{empty set}}, $A \cap B = \varnothing$. We also express this fact by saying that $A$ and $B$ are \textbf{\uwave{disjoint}}.

\paragraph{The Difference of Two Sets} There is one other operation on sets that is occasionally useful. it is the \textbf{\uwave{difference}} of two sets, denoted by $A - B$ or $A \backslash B$. It is defined as the set consisting of those elements of $A$ that are not in $B$. Formally, we define $A - B = A \backslash B = \{x: x\in A \text{ and } x \not\in B\}$

% Definition of circles
\def\firstcircle{(0,0) circle (1.1cm)}
\def\secondcircle{(0:1.5cm) circle (1.1cm)}

\colorlet{circle edge}{red!50}
\colorlet{circle area}{red!20}

\tikzset{filled/.style={fill=circle area, draw=circle edge, thick},
    outline/.style={draw=circle edge, thick}}

\setlength{\parskip}{5mm}

\begin{figure}[htp]
    \centering
    \begin{subfigure}[b]{0.25\textwidth}
    \centering
    \begin{tikzpicture}
        \begin{scope}
            \clip \firstcircle;
            \fill[filled] \secondcircle;
        \end{scope}
        \draw[outline] \firstcircle node {$A$};
        \draw[outline] \secondcircle node {$B$};
        \node[anchor=south] at (current bounding box.north) {$A \cap B$};
    \end{tikzpicture}
    \end{subfigure}
    \begin{subfigure}[b]{0.25\textwidth}
    \centering
    % Set A or B
    \begin{tikzpicture}
        \draw[filled] \firstcircle node {$A$}
                      \secondcircle node {$B$};
        \node[anchor=south] at (current bounding box.north) {$A \cup B$};
    \end{tikzpicture}
    \end{subfigure}
    \begin{subfigure}[b]{0.25\textwidth}
    \centering
    % Set A but not B
    \begin{tikzpicture}
        \begin{scope}
            \clip \firstcircle;
            \draw[filled, even odd rule] \firstcircle node {$A$}
                                         \secondcircle;
        \end{scope}
        \draw[outline] \firstcircle
                       \secondcircle node {$B$};
        \node[anchor=south] at (current bounding box.north) {$A \backslash B$};
    \end{tikzpicture}
    \end{subfigure}
    \caption{Operations in Sets}
    \label{fig:set_operations}
\end{figure}

\subsubsection{Rules of Set Theory}

Given several sets, one may form new sets by applying the set-theoretic operations to them. As in algebra, one uses parentheses to indicate in what order the operations are to be performed. For example, $A \cup (B \cap C)$ denotes the union of the two set $A$ and $B \cap C$, while $(A \cup B) \cap C$ denotes the intersection of the two sets $A \cup B$ and $C$. Sometimes different combinations of operations lead to the same set, when that happens, one has a rule of set theory (which can be thought of as a ``distributive law" for the operations $\cup$ and $\cap$):
\newpage
\begin{enumerate}[itemsep=0pt,topsep=0pt]
    \item $A \cap (B \cup C) = (A \cap B) \cup (A \cap C)$
    \item $A \cup (B \cap C) = (A \cup B) \cap (A \cup C)$
\end{enumerate}

and \textbf{\uwave{DeMorgan's Law}}:
\begin{enumerate}[itemsep=0pt,topsep=0pt]
    \item $A \backslash (B\cup C) = (A \backslash B) \cap (A \backslash C)$
    \item $A \backslash (B\cap C) = (A \backslash B) \cup (A \backslash C)$
\end{enumerate}

\subsubsection{Collections of Sets}

The elements belonging to a set may be of any sort. One can consider the set of all even integers, and the set of all blue-eyed people in the world. Some of these are of limited mathematical interest. Can the set be elements of another sets? The answer is yes. We now have another way to form new sets from old ones. Given a set $A$, we can consider sets whose elements are subsets of $A$. In particular, \uwave{we can consider the set of all subsets of $A$.} This set is sometimes denoted by the symbol $\mathcal{P}(A)$ and is called the \textbf{\uwave{power set}} of $A$.\footnote{Why we can the collection of $A$'s subset a power set? If $A= \{a, b, c\}$, then $\mathcal{P}(A) = \{\varnothing, \{a\}, \{b\}, \{c\}, \{a, b\}, \{a, c\}, \{b, c\}, \{a, b, c\}\}$ contains $8$ elements, and $2^3 = 8$, thus, it is named as \textbf{power} set.}

When we have \uline{a set whose elements are sets}, we shall often refer to it as a \textbf{\uwave{collection of sets}} and denoted it by a script letter such as $\mathcal{A}$ or $\mathcal{B}$. 

Thus, if $A = \{a, b, c\}$, then $a$ is a element of $A$ ($a \in A$), $\{a\}$ is a subset of $A$ ($\{a\} \subset A$), $\{a\}$ is a element of $\mathcal{P}(A)$($\{a\} \in \mathcal{P}(A)$).

\paragraph{Arbitrary Union and Intersections} We have already defined what we mean by the union and the intersection of two sets. There is no reason to limit ourselves to just two sets, for we can just as well form the union and intersection of arbitrarily many sets.

\begin{enumerate}[itemsep=0pt]
    \item Given a collection $\mathcal{A}$, the \textbf{\uwave{union}} of the elements of $\mathcal{A}$ is: $\bigcup_{A \in \mathcal{A}} A = \{x: x \in A \text{ for at least one } A \in \mathcal{A}\}$
    \item Given a collection $\mathcal{A}$, the \textbf{\uwave{intersection}} of the elements of $\mathcal{A}$ is: $\bigcap_{A \in \mathcal{A}} A = \{x: x \in A \text{ for every } A \in \mathcal{A}\}$
\end{enumerate}

\subsubsection{Cartesian products}

There is yet another way of forming new sets from old ones; it involves the notion of an ``ordered pair" of objects. When we are considering a point $(x, y) \in \mathbb{R}^2$, we are actually defining a ``ordered pair" of real number $(x, y)$. (In a more sophisticated treatment of geometry, the plane, is more likely to be defined as the set of all ordered pairs of real numbers)

The notion of ordered pair carries over to general sets. Given two sets $A$ and $B$m we define their \textbf{\uwave{cartesian product}} $A \times B$ to be the set of all ordered pair $(a, b), \forall a \in A, \forall b \in B$. Formally, $A \times B = \{(a, b): a \in A, b \in B\}$.


\subsection{Functions}

A function is usually thought of as a \uwave{rule} that assigns to each element of a set $A$, an element of a set $B$, In calculus, a function is often given by a simple formula such as $f(x) = 3x^2 + 2$ or perhaps by a more complicated formula such as $f(x) = \sum_{i=1}^{\infty}x^i$.

\begin{definition}[rule of assignment]
\textbf{\uwave{A rule of assignment}} is a subset $r$ of the Cartesian product $C \times D$ of two sets, having the property that each element of $C$ appears as the firs coordinate of at most one ordered pair belonging to $r$.
\end{definition}

Thus, a subset $r$ of $C \times D$ is a rule of assignment if $[(c, d) \in r \text{ and } [c^{\prime}, d^{\prime}] \in r] \Rightarrow [d = d^{\prime}]$. We think of $r$ as a way of assigning, to the element $c$ of $C$, the element $d$ of $D$ for which $(c, d) \in r$.

Given a rule of assignment $r$, the \textbf{\uwave{domain}} of $r$ is defined to be the subset of $C$ consisting of all first coordinates of elements of $r$, and the \textbf{\uwave{image set}} of $r$ is defined as the subset of $D$ consisting of all second coordinates of elements of $r$. Formally:
\begin{enumerate}[itemsep=0pt]
    \item domain $r= \{c: \text{ these exists } d \in D \text{ such that } (c, d) \ in r\}$
    \item image $r = \{d: \text{ these exists } c \in C \text{ such that } (c, d) \ in r\}$
\end{enumerate}

Note that given a rule of assignment $r$, its domain and image are entirely determined.

\begin{definition}[function]
A \textbf{\uwave{function}} $f$ is a rule of assignment $r$, together with a set $B$ that contains the image set of $r$. The domain $A$ of the rule $r$ is also called the \textbf{domain} of function $f$; the image set of $r$ is also called the \textbf{image set} of $f$; and the set $B$ is called the \textbf{range} of $f$.\footnote{Analysts are apt to use the word ``range" to denote what we have called the ``image set" of $f$. They avoid giving the set $B$ a name.}

If $f$ is a function having domain $A$ and range $B$, we express this fact by writing $f: A \mapsto B$, which is read ``$f$ is a function from $A$ to $B$.", or ``$f$ maps $A$ to $B$".
\end{definition}

\begin{definition}[resitriction]
If $f: A \mapsto B$ and if $A_0$ is a subset of $A$, we define the \textbf{\uwave{restriction}} of $f$ to $A_0$ to be the function mapping $A_0$ into $B$ whose rule is $\{(a, f(a)) | a \in A_0\}$
\end{definition}

\begin{definition}[composite]
Given functions $f:A \mapsto B$ and $g: B \mapsto C$, we define the \textbf{\uwave{composite}} $g \circ f$ of $f$ and $g$ as the function $g \circ f: A \mapsto C$ defined by the equation $(g \circ f)(a) = g(f(a)), a \in A$. Formally, $g \circ f: A \mapsto C$ is the function whose rule is $\{(a, c), \text{ for some } b\in B, f(a) = b \text{ and } g(b) = c\}$.
\end{definition}

Considering two physical movement: rotation and translation, if we first rotate a object and then do the translation, it is a kind of composite function $g \circ f$.

\begin{figure}[htp]
    \centering
    \begin{tikzpicture}
        % plots
        \draw plot [smooth cycle] coordinates {(1.0,.1)(1.5,.2)(2.8,.5)(2.9,1.5)(2.8,2.8)(1.4,2.5)(0.5,0.5)} 
                node[circle, fill=black, inner sep=1pt, label=below:$x$] (u) at (1.8,1.8) {};
                \draw[fill=cyan!80, dashed, fill opacity=0.4] (1.8,1.8) circle (4pt);
        \draw plot [smooth cycle] coordinates {(5,0.25) (6,0.35) (6.5, 0.2) (7,0.5) (7,1.65) (6.5,2.75) (5.8,2.75) (5.3,1.45) (4.8,0.85) } 
                node[circle, fill=black, inner sep=1pt, label=below:$f(x)$] (x) at (6,1.7) {};
                \draw[fill=orange!70, dashed, fill opacity=0.4] (6,1.7) circle (4pt);
        \draw plot [smooth cycle] coordinates {(8,0.25) (9,0.15) (9.5, 0.2) (10,0.5) (9.5,1.65) (8.8,2.95) (8.4,2.45) (8.2,2.55) (7.8,0.95) } 
                node[circle, fill=black, inner sep=1pt, label=below:$g(x)$] (z) at (9.1,1.9) {};
                \draw[fill=orange!70, dashed, fill opacity=0.4] (9.1,1.9) circle (4pt);
        \draw[-{Straight Barb[length=5pt,width=5pt]}, dashed] (u) edge[out=-20, in=160] node[above] {$f(\cdot)$} (x);
                \draw[-{Straight Barb[length=5pt,width=5pt]}, dashed] (x) edge[out=-20, in=160] node[above] {$g(\cdot)$} (z);
        \end{tikzpicture} 
    \caption{$g \circ f$}
    \label{fig:composite_map}
\end{figure}

\begin{definition}[injective, surjective, bijective]
A function $f: A \mapsto B$ is said to be \textbf{\uwave{injective (or one-to one)}} if for each pair of distinct points of $A$, their images under $f$ are distinct. It is said to be \textbf{\uwave{surjective}} (or $f$ is said to map $A$ \textbf{onto} $B$) if every element of $B$ is the image of some element of $A$ under the function $f$. If $f$ is both \uwave{injective and surjective}, it is said to be \textbf{\uwave{bijective}} (or is called a \uwave{ono-to-one correspondence}).
\end{definition}

More formally, $f$ is \uwave{injective} if $[f(a) = f(a^{\prime})] \Rightarrow [a = a^{\prime}]$; $f$ is \uwave{surjective} if $[b \in B] \Rightarrow [b \in f(a) \text{ for at least one } a\in A]$

If $f$ is bijective, there exists a function from $B$ to $A$ is called \textbf{\uwave{inverse}} of $f$, denoted by $f^{-1}$ and is defined by letting $f^{-1}(b), b \in B$ be that unique element $a \in A$ for which $f(a) = b$.

\begin{lemmad}
Let $f: A \mapsto B$. If there are functions $g: B \mapsto A$ and $g: B \mapsto A$ such that $g(f(a)) = a$ for every $a \in A$ and $f(h(b)) = b$ for $b \in B$, then $f$ is a bijective and $g = h = f^{-1}$.
\end{lemmad}

\begin{definition}[image and preimage of a subset]
Let $f: A \mapsto B$. If $A_0$ is a subset of $A$, we denote by $f(A_0)$ the set of all images of point of $A_0$ under the function $f$; this set is called the \textbf{\uwave{image of $A_0$ under $f$}}. Formally, $f(A_0) = \{b: b = f(a) \text{ for at least one } a\in A_0\}$. O the other hand, if $B_0$ is a subset of $B$, we denote by $f^{-1}(B_0)$ the set of all element of $A$ whose images under $f$ lies in $B_0$; it is called the \textbf{\uwave{preimage}} of $B_0$ under $f$. Formally, $f^{-1}(B_0) = \{a: f(a) \in B_0\}$.
\end{definition}

Note that if $f:A \mapsto B$ is a bijective and $B_0 \subset B$, then $B_0$ is the \uwave{image of $f$} and the \uwave{preimage of $f^{-1}$}. 





\section{Topological Spaces and Continuous Functions}

\section{Connectedness and Compactness}



% part 2
\part{Metric Space}
\include{context/part2}


\part{Banach and Hilbert Spaces}
\include{context/part3}

%part 4
\part{Banache Algebra}
\include{conext/part4}

\newpage
\bibliographystyle{unsrt}
\bibliography{sample}

%\newpage

%\newpage

\end{document}