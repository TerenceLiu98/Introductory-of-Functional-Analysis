\begin{abstract}
\noindent Functional analysis is a rich subject because it combines two large branches of mathematic: the \textit{topological} branch concerns itself with convergence, continuity,
connectivity, boundedness, etc.; and the \textit{algebraic} branch concerns itself with operations, groups, rings, vectors, etc. Since this is the introductory course on Functional Analysis, in this course, we mainly focus on these parts:

    \begin{enumerate}
        \item Some Preliminaries in Analysis
        \item Metric space
        \item Norm Space and Banach Space
        \item Linear vector space, Linear Operator and Dual Space
        \item Inner Product space and Hilbert Space
    \end{enumerate}
    
\noindent Here are some reference books and lecture notes:
    
    \begin{enumerate}
        \item Books\footnote{The highlighted implies ``highly recommended"}:
        \begin{enumerate}
            \item \textit{Tong Sun - Functional Analysis \cite{sun2010functional}}
            \item \textit{Yisheng Huang - Functional Analysis An Introduction Second Edition \cite{huang2009functional}}
            \item \textit{E. Kreyszig - Introductory Functional Analysis with Application \cite{kreyszig1978introductory}}
            \item \textit{Joseph Muscat - \textbf{Functional Analysis - An Introduction to metric Spaces, Hilbert Spaces, and Banach Algebras}\cite{muscat2014functional}}
            \item \textit{Philippe G. Ciarlet - \textbf{Linear and Nonlinear Functional Analysis with Applications}\cite{ciarlet2013linear}}
            \item \textit{Terence Tao - \textbf{Analysis}\cite{tao2009analysis}}
            \item \textit{Matthew A. Pons - \textbf{Real Analysis for the Undergraduate} \cite{pons2014real}}
        \end{enumerate}
        \item Notes:
        \begin{enumerate}
            \item \href{http://www.math.utah.edu/~tan/notes.html}{Mr. Chee Han Tan's note (PhD Candidate in Utah University)}
            \item \href{http://math.wsu.edu/students/jstreipel/notes/functionalanalysis.pdf}{Mr. Jakob Streipel's Note (Doctoral student in Washington State University)} 
            \item Dr.Man Wa HUI's Lecture note (BNU-HKBU UIC Course MATH7110 - Functional Analysis)
            \item \href{https://www.math.cuhk.edu.hk/course_builder/2122/math4010/Functional\%20analysis\%202021-22a.pdf}{Prof. Chi Wai LEUNG's Lecture note} (The Chinese University of Hong Kong Course MATH4010 - Functional Analysis)
            \item \href{https://www.math.ucdavis.edu/~hunter/m201a_16/}{John K.Hunter - Math 201A: Analysis, Fall 2016}
        \end{enumerate}
        %\item Videos:
        %\begin{enumerate}
        %    \item \href{https://www.youtube.com/playlist?list=PLBh2i93oe2qsGKD%OsuVVw-OCAfprrnGfr}{\textit{Functional Analysis} by The Right Side of %Mathematics}
        %\end{enumerate}
    \end{enumerate}
    
However, most of the Statistics undergraduate in BNU-HKBU United International College did not take the course of real analysis, only basic Multivariate Calculus, thus, there are multiple preliminary's knowledge missing. In this note, I am trying to fill the gap by covering the missing chapters and I highly recommend those students who enrol this course spend some time on Mathematical Analysis and Real Analysis, also, take some time to do the revision on Linear Algebra.

As a guide, the notes and exercises have been marked as follows:

\begin{enumerate}
    \item[$\blacktriangleright$] refers to important notes and results, also \uwave{wave}, \uline{underline} or \textbf{bold}.
    \item[$*$] more advanced or difficult exercises that can be skipped on a first reading;
    \item[$\diamondsuit$] side remarks that can be skipped without losing any essential ideas
\end{enumerate}

\noindent``数学当中最麻烦的事情就是显然,我觉得显然,你觉得不显然这就是最麻烦的事情了” - ``Trivial is the biggest problem in mathematics". \footnote{from article \href{https://zhuanlan.zhihu.com/p/38029151}{``选择公理与 Zorn 引理" by Diet Meat, posted in Zhihu}}

\end{abstract}