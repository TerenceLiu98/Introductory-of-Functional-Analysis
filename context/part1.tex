\section{Set Theory and Logic}

\subsection{Fundamental Concepts}\label{sec:fundamental_concepts}

Here we introduce the ideas of set theory, and establish the basic terminology and notation. We also discuss some points of elementary logic.

\subsubsection{Basic Notation}\label{sec:basic_notation}

Commonly we shall use capital letters $A, B, \dots$ to denote \textbf{\uwave{set}}, and lowercase letters $a, b, \dots,$ to denote the \textbf{\uwave{elements}} belonging to these sets. If an element $a$ belongs to a set $A$, we  express this fact by the notation $a \in A$; for the contradiction, if $a$ does not belong to $A$, we express this fact by writing: $a \not\in A$. 

The equality symbol $=$ is used throughout this note to mean \uwave{logical identity}. Thus, when we say $a = b$, we mean $a$ and $b$ are symbols for the same element. Similarly, if $a$ and $b$ are different elements, we write $a \neq b$.

We say that $A$ is a \textbf{\uwave{subset}} if $B$ if every element of $A$ is also an element of $B$, denoted by $A \subset B$; ($\forall x \in A, x \in B$) Noting in this definition requires $A$ being different from $B$; in fact, \uline{if $A = B$, it is true that both $A \subset B$ and $B \subset A$.} If $A \subset B$ and $A$ is different from $B$, then, we say that $A$ is \textbf{\uwave{proper subset}} of $B$, denoted by $A \subseteq B$.\footnote{The relations $\subset$ and $\subseteq$ are called \textbf{\uwave{inclusion} and \uwave{proper inclusion}}, respectively.}

\subsubsection{Operations with Sets}\label{sec:operations_with_sets}

\paragraph{The Union of Sets and the meaning of "or"} Given two sets $A$ and $B$, one can form a set from them that consists of all the elements of $A$ together with all elements of $B$. This set is called \textbf{\uwave{the union}} of $A$ and $B$, is denoted by $A \cup B$. Formally, we define $A \cup B = \{x: x\in A, x \in B\}$. The union of two sets, means the elements of the new sets is from $A$, or from $B$, or from both $A$ and $B$.

\paragraph{The Intersection of Sets, and the Empty Set} Given sets $A$ and $B$, another way one can form a set is to take the common part of $A$ and $B$. This set is called the \textbf{\uwave{intersection}} of $A$ and $B$ and is denoted by $A \cap B$. Formally, we define $A \cap B = \{x: x\in A \text{ and } x \in B\}$. If the set $A$ and $B$ has no common element, then we call the intersection of $A$ and $B$ is a \textbf{\uwave{empty set}}, $A \cap B = \varnothing$. We also express this fact by saying that $A$ and $B$ are \textbf{\uwave{disjoint}}.

\paragraph{The Difference of Two Sets} There is one other operation on sets that is occasionally useful. it is the \textbf{\uwave{difference}} of two sets, denoted by $A - B$ or $A \backslash B$. It is defined as the set consisting of those elements of $A$ that are not in $B$. Formally, we define $A - B = A \backslash B = \{x: x\in A \text{ and } x \not\in B\}$

% Definition of circles
\def\firstcircle{(0,0) circle (1.1cm)}
\def\secondcircle{(0:1.5cm) circle (1.1cm)}

\colorlet{circle edge}{red!50}
\colorlet{circle area}{red!20}

\tikzset{filled/.style={fill=circle area, draw=circle edge, thick},
    outline/.style={draw=circle edge, thick}}

\setlength{\parskip}{5mm}

\begin{figure}[htp]
    \centering
    \begin{subfigure}[b]{0.25\textwidth}
    \centering
    \begin{tikzpicture}
        \begin{scope}
            \clip \firstcircle;
            \fill[filled] \secondcircle;
        \end{scope}
        \draw[outline] \firstcircle node {$A$};
        \draw[outline] \secondcircle node {$B$};
        \node[anchor=south] at (current bounding box.north) {$A \cap B$};
    \end{tikzpicture}
    \end{subfigure}
    \begin{subfigure}[b]{0.25\textwidth}
    \centering
    % Set A or B
    \begin{tikzpicture}
        \draw[filled] \firstcircle node {$A$}
                      \secondcircle node {$B$};
        \node[anchor=south] at (current bounding box.north) {$A \cup B$};
    \end{tikzpicture}
    \end{subfigure}
    \begin{subfigure}[b]{0.25\textwidth}
    \centering
    % Set A but not B
    \begin{tikzpicture}
        \begin{scope}
            \clip \firstcircle;
            \draw[filled, even odd rule] \firstcircle node {$A$}
                                         \secondcircle;
        \end{scope}
        \draw[outline] \firstcircle
                       \secondcircle node {$B$};
        \node[anchor=south] at (current bounding box.north) {$A \backslash B$};
    \end{tikzpicture}
    \end{subfigure}
    \caption{Operations in Sets}
    \label{fig:set_operations}
\end{figure}

\subsubsection{Rules of Set Theory}\label{sec:rules_of_set_theory}

Given several sets, one may form new sets by applying the set-theoretic operations to them. As in algebra, one uses parentheses to indicate in what order the operations are to be performed. For example, $A \cup (B \cap C)$ denotes the union of the two set $A$ and $B \cap C$, while $(A \cup B) \cap C$ denotes the intersection of the two sets $A \cup B$ and $C$. Sometimes different combinations of operations lead to the same set, when that happens, one has a rule of set theory (which can be thought of as a ``distributive law" for the operations $\cup$ and $\cap$):
\newpage
\begin{enumerate}[itemsep=0pt,topsep=0pt]
    \item $A \cap (B \cup C) = (A \cap B) \cup (A \cap C)$
    \item $A \cup (B \cap C) = (A \cup B) \cap (A \cup C)$
\end{enumerate}

and \textbf{\uwave{DeMorgan's Law}}:
\begin{enumerate}[itemsep=0pt,topsep=0pt]
    \item $A \backslash (B\cup C) = (A \backslash B) \cap (A \backslash C)$
    \item $A \backslash (B\cap C) = (A \backslash B) \cup (A \backslash C)$
\end{enumerate}

\subsubsection{Collections of Sets}\label{sec:collections_of_sets}

The elements belonging to a set may be of any sort. One can consider the set of all even integers, and the set of all blue-eyed people in the world. Some of these are of limited mathematical interest. Can the set be elements of another sets? The answer is yes. We now have another way to form new sets from old ones. Given a set $A$, we can consider sets whose elements are subsets of $A$. In particular, \uwave{we can consider the set of all subsets of $A$.} This set is sometimes denoted by the symbol $\mathcal{P}(A)$ and is called the \textbf{\uwave{power set}} of $A$.\footnote{Why we can the collection of $A$'s subset a power set? If $A= \{a, b, c\}$, then $\mathcal{P}(A) = \{\varnothing, \{a\}, \{b\}, \{c\}, \{a, b\}, \{a, c\}, \{b, c\}, \{a, b, c\}\}$ contains $8$ elements, and $2^3 = 8$, thus, it is named as \textbf{power} set.}

When we have \uline{a set whose elements are sets}, we shall often refer to it as a \textbf{\uwave{collection of sets}} and denoted it by a script letter such as $\mathcal{A}$ or $\mathcal{B}$. 

Thus, if $A = \{a, b, c\}$, then $a$ is a element of $A$ ($a \in A$), $\{a\}$ is a subset of $A$ ($\{a\} \subset A$), $\{a\}$ is a element of $\mathcal{P}(A)$($\{a\} \in \mathcal{P}(A)$).

\paragraph{Arbitrary Union and Intersections} We have already defined what we mean by the union and the intersection of two sets. There is no reason to limit ourselves to just two sets, for we can just as well form the union and intersection of arbitrarily many sets.

\begin{enumerate}[itemsep=0pt]
    \item Given a collection $\mathcal{A}$, the \textbf{\uwave{union}} of the elements of $\mathcal{A}$ is: $\bigcup_{A \in \mathcal{A}} A = \{x: x \in A \text{ for at least one } A \in \mathcal{A}\}$
    \item Given a collection $\mathcal{A}$, the \textbf{\uwave{intersection}} of the elements of $\mathcal{A}$ is: $\bigcap_{A \in \mathcal{A}} A = \{x: x \in A \text{ for every } A \in \mathcal{A}\}$
\end{enumerate}

\subsubsection{Cartesian products}\label{sec:cartesian_products}

There is yet another way of forming new sets from old ones; it involves the notion of an ``ordered pair" of objects. When we are considering a point $(x, y) \in \R^2$, we are actually defining a ``ordered pair" of real number $(x, y)$. (In a more sophisticated treatment of geometry, the plane, is more likely to be defined as the set of all ordered pairs of real numbers)

The notion of ordered pair carries over to general sets. Given two sets $A$ and $B$m we define their \textbf{\uwave{cartesian product}} $A \times B$ to be the set of all ordered pair $(a, b), \forall a \in A, \forall b \in B$. Formally, $A \times B = \{(a, b): a \in A, b \in B\}$.


\subsection{Functions}\label{sec:functions}

A function is usually thought of as a \uwave{rule} that assigns to each element of a set $A$, an element of a set $B$, In calculus, a function is often given by a simple formula such as $f(x) = 3x^2 + 2$ or perhaps by a more complicated formula such as $f(x) = \sum_{i=1}^{\infty}x^i$.

\begin{definition}[rule of assignment]
\textbf{\uwave{A rule of assignment}} is a subset $r$ of the Cartesian product $C \times D$ of two sets, having the property that each element of $C$ appears as the firs coordinate of at most one ordered pair belonging to $r$.
\end{definition}

Thus, a subset $r$ of $C \times D$ is a rule of assignment if $[(c, d) \in r \text{ and } [c^{\prime}, d^{\prime}] \in r] \Rightarrow [d = d^{\prime}]$. We think of $r$ as a way of assigning, to the element $c$ of $C$, the element $d$ of $D$ for which $(c, d) \in r$.

Given a rule of assignment $r$, the \textbf{\uwave{domain}} of $r$ is defined to be the subset of $C$ consisting of all first coordinates of elements of $r$, and the \textbf{\uwave{image set}} of $r$ is defined as the subset of $D$ consisting of all second coordinates of elements of $r$. Formally:
\begin{enumerate}[itemsep=0pt]
    \item domain $r= \{c: \text{ these exists } d \in D \text{ such that } (c, d) \ in r\}$
    \item image $r = \{d: \text{ these exists } c \in C \text{ such that } (c, d) \ in r\}$
\end{enumerate}

Note that given a rule of assignment $r$, its domain and image are entirely determined.

\begin{definition}[function]
A \textbf{\uwave{function}} $f$ is a rule of assignment $r$, together with a set $B$ that contains the image set of $r$. The domain $A$ of the rule $r$ is also called the \textbf{domain} of function $f$; the image set of $r$ is also called the \textbf{image set} of $f$; and the set $B$ is called the \textbf{range} of $f$.\footnote{Analysts are apt to use the word ``range" to denote what we have called the ``image set" of $f$. They avoid giving the set $B$ a name.}

If $f$ is a function having domain $A$ and range $B$, we express this fact by writing $f: A \mapsto B$, which is read ``$f$ is a function from $A$ to $B$.", or ``$f$ maps $A$ to $B$".
\end{definition}

\begin{definition}[resitriction]
If $f: A \mapsto B$ and if $A_0$ is a subset of $A$, we define the \textbf{\uwave{restriction}} of $f$ to $A_0$ to be the function mapping $A_0$ into $B$ whose rule is $\{(a, f(a)) | a \in A_0\}$
\end{definition}

\begin{definition}[composite]
Given functions $f:A \mapsto B$ and $g: B \mapsto C$, we define the \textbf{\uwave{composite}} $g \circ f$ of $f$ and $g$ as the function $g \circ f: A \mapsto C$ defined by the equation $(g \circ f)(a) = g(f(a)), a \in A$. Formally, $g \circ f: A \mapsto C$ is the function whose rule is $\{(a, c), \text{ for some } b\in B, f(a) = b \text{ and } g(b) = c\}$.
\end{definition}

Considering two physical movement: rotation and translation, if we first rotate a object and then do the translation, it is a kind of composite function $g \circ f$.

\begin{figure}[htp]
    \centering
    \begin{tikzpicture}
        % plots
        \draw plot [smooth cycle] coordinates {(1.0,.1)(1.5,.2)(2.8,.5)(2.9,1.5)(2.8,2.8)(1.4,2.5)(0.5,0.5)} 
                node[circle, fill=black, inner sep=1pt, label=below:$x$] (u) at (1.8,1.8) {};
                \draw[fill=cyan!80, dashed, fill opacity=0.4] (1.8,1.8) circle (4pt);
        \draw plot [smooth cycle] coordinates {(5,0.25) (6,0.35) (6.5, 0.2) (7,0.5) (7,1.65) (6.5,2.75) (5.8,2.75) (5.3,1.45) (4.8,0.85) } 
                node[circle, fill=black, inner sep=1pt, label=below:$f(x)$] (x) at (6,1.7) {};
                \draw[fill=orange!70, dashed, fill opacity=0.4] (6,1.7) circle (4pt);
        \draw plot [smooth cycle] coordinates {(8,0.25) (9,0.15) (9.5, 0.2) (10,0.5) (9.5,1.65) (8.8,2.95) (8.4,2.45) (8.2,2.55) (7.8,0.95) } 
                node[circle, fill=black, inner sep=1pt, label=below:$g(x)$] (z) at (9.1,1.9) {};
                \draw[fill=orange!70, dashed, fill opacity=0.4] (9.1,1.9) circle (4pt);
        \draw[-{Straight Barb[length=5pt,width=5pt]}, dashed] (u) edge[out=-20, in=160] node[above] {$f(\cdot)$} (x);
                \draw[-{Straight Barb[length=5pt,width=5pt]}, dashed] (x) edge[out=-20, in=160] node[above] {$g(\cdot)$} (z);
        \end{tikzpicture} 
    \caption{$g \circ f$}
    \label{fig:composite_map}
\end{figure}

\begin{definition}[injective, surjective, bijective]
A function $f: A \mapsto B$ is said to be \textbf{\uwave{injective (or one-to one)}} if for each pair of distinct points of $A$, their images under $f$ are distinct. It is said to be \textbf{\uwave{surjective}} (or $f$ is said to map $A$ \textbf{onto} $B$) if every element of $B$ is the image of some element of $A$ under the function $f$. If $f$ is both \uwave{injective and surjective}, it is said to be \textbf{\uwave{bijective}} (or is called a \uwave{ono-to-one correspondence}).
\end{definition}

More formally, $f$ is \uwave{injective} if $[f(a) = f(a^{\prime})] \Rightarrow [a = a^{\prime}]$; $f$ is \uwave{surjective} if $[b \in B] \Rightarrow [b \in f(a) \text{ for at least one } a\in A]$

If $f$ is bijective, there exists a function from $B$ to $A$ is called \textbf{\uwave{inverse}} of $f$, denoted by $f^{-1}$ and is defined by letting $f^{-1}(b), b \in B$ be that unique element $a \in A$ for which $f(a) = b$.

\begin{lemmad}
Let $f: A \mapsto B$. If there are functions $g: B \mapsto A$ and $g: B \mapsto A$ such that $g(f(a)) = a$ for every $a \in A$ and $f(h(b)) = b$ for $b \in B$, then $f$ is a bijective and $g = h = f^{-1}$.
\end{lemmad}

\begin{definition}[image and preimage of a subset]
Let $f: A \mapsto B$. If $A_0$ is a subset of $A$, we denote by $f(A_0)$ the set of all images of point of $A_0$ under the function $f$; this set is called the \textbf{\uwave{image of $A_0$ under $f$}}. Formally, $f(A_0) = \{b: b = f(a) \text{ for at least one } a\in A_0\}$. O the other hand, if $B_0$ is a subset of $B$, we denote by $f^{-1}(B_0)$ the set of all element of $A$ whose images under $f$ lies in $B_0$; it is called the \textbf{\uwave{preimage}} of $B_0$ under $f$. Formally, $f^{-1}(B_0) = \{a: f(a) \in B_0\}$.
\end{definition}

Note that if $f:A \mapsto B$ is a bijective and $B_0 \subset B$, then $B_0$ is the \uwave{image of $f$} and the \uwave{preimage of $f^{-1}$}. 

\subsection{Relations}

A more general concept of functions is the \textbf{relation}.

\begin{definition}[relation]
A \textbf{\uwave{relation}} on a set $A$ is a subset $C$ of the cartesian product $A \times A$.
\end{definition}

If $C$ is a relation on $A$, we use the notation $xCy$ to mean the same thing as $(x, y) \in C$. and we read it as ``$x$ in the relation $C$ to $y$".

A rule of assignment $r$ for a function $f: A \mapsto A$ is also a subset of $A \times A$. But it is a subset of a very special kind: namely, one such that each element of $A$ appears as the first coordinate of an element of $r$ exactly once. Any subset of $A \times A$ is a relation on $A$.

\paragraph{Equivalence Relations and Partitions} An \textbf{\uwave{equivalence relation}} on a st $A$ is a relation $C$ on $A$ having the following three properties:
\begin{enumerate}[itemsep=0pt]
    \item (Reflexivity) $xCx$ for every $x \in A$
    \item (Symmetry) If $xCy$, then $yCx$
    \item (Transitivity) If $xCy$ and $yCz$, then $xCz$
\end{enumerate}

There is no reason one must use a capital letter  - or indeed a letter of any sort - to denote a relation, even though it is a (ordered-pair) set. Another symbol will do just as well, One symbol that is frequently used to denoted an equivalence relation become 
\begin{enumerate}[itemsep=0pt]
    \item $x \sim x$ for every $x \in A$
    \item If $x \sim y$, then $y \sim x$
    \item If $x \sim y$ and $y \sim z$, then $x \sim z$
\end{enumerate}

Given an equivalence relation $\sim$ on $A$ and an element $x$ of $A$, we define a certain subset $E$ of $A$, called the \textbf{\uwave{equivalence class}} determined by $x$, by the equation $E = \{y: y \sim x\}$. Notice that the equivalence class $E$ determined by $x$ contains $x$, since $x \sim x$, The set $E$ have the following property:

\begin{lemmad}
Two equivalence classes $E$ and $E^{\prime}$ are either disjoint or equal.
\end{lemmad}
\begin{proof}
Let $E$ be the equivalence class determined by $x$, correspondingly $E^{\prime}$ is the equivalence class determined by $x^{\prime}$. First we assume $E \cap E^{\prime} = \varnothing$, thus, they are disjoint; then we consider $E \cap E^{\prime} \neq \varnothing$, and assume the interception contains an element $y$. Since $y \in E$, thus, $y \sim x_i, \forall x_i \in E$. Similarly, $y \in E^{\prime}$, thus, $y \sim x^{\prime}_i, \forall x^{\prime}_i \in E^{\prime}$. With the property of transitivity, we can get $x \sim x^{\prime}$, thus, $E$ and $E^{\prime}$ are equal.
\end{proof}

Given an equivalence relation on a set $A$, let us denote by $\mathcal{E}$ the collection of all equivalence classes determined by this relation. The proceeding lemma shows that distinct element of $\mathcal{E}$ are disjoint. Furthermore, the union of the elements of $\mathcal{E}$ equals all of $A$ because every element of $A$ belongs to an equivalence class. The collection $\mathcal{E}$ is a particular example of what is called a \uwave{partition} of $A$:

\begin{definition}[partition]
A \textbf{\uwave{partition}} of a set $A$ is a collection of disjoint nonempty subset of $A$ whose union is all of $A$. Formally, $\bigcup_{E \in \mathcal{E}} E = A$.
\end{definition}

Studying equivalence relations on a set $A$ and studying partitions of $A$ are really the same thing. Given any partition $\mathcal{D}$ of $A$, there is exactly one equivalence relation on $A$ from which it is derived.

\paragraph{Order Relations} A relation $C$ on a set $A$ is called an \textbf{\uwave{order relation}} (or a \uwave{simple order}, or \uwave{linear order}) if it has the following properties:
\begin{enumerate}[itemsep=0pt]
    \item (Comparability) For every $x$ and $y$ in $]A$ for which $x \not y$, either $xCy$ or $yCx$.
    \item (Nonreflexivity) For no $x$ in $A$ does the relation $xCx$ hold.
    \item (Transitivity) If $xCy$ and $yCz$, then $xCz$
\end{enumerate}

Note that the ``Comparability" does not by itself exclude the possibility that for some pair of elements $x$ and $y$ of $A$, both the relations $xCy$ and $yCx$ holds. But the ``Nonreflexivity" and ``Transitivity" combined do exclude this possibility; for if both $xCy$ and $yCx$ held, transitivity would imply that $xCx$, contradicting nonreflexivity.

As the tilde, $\sim$, is the generic symbol for an equivalence relation, the ``less than" symbol, $<$, is commonly used to denote an order relation. Stated in this notation, we can rewrite the above three properties:
\begin{enumerate}[itemsep=0pt]
    \item (Comparability) If $x \neq y$, then either $x < y$ or $y < x$
    \item (Nonreflexivity) If $x < y$, then $x \neq y$
    \item (Transitivity) If $x < y$ and $y < z$, then $x < z$
\end{enumerate}

\begin{definition}
If $(X, <)$ is a set with order relation, and if $a < b$, we use notation $(a, b)$ to denote the set $\{x: a < x < b\}$; it is called an \textbf{\uwave{open interval}} in $X$. If this set is empty, we call $a$ the \textbf{\uwave{immedia predecessor}} of $b$, and we call $b$ the \textbf{\uwave{immedia predecessor}} of $a$.
\end{definition}

\begin{definition}
Suppose that $A$ and $B$ are two sets with order relations respectively. We say that two sets have the same \textbf{\uwave{order type}} if there is a bijective correspondence between them that preserves order; that is, if there exists a bijective function $f:A \mapsto B$ such that $a_1 <_A a_2 \Rightarrow f(a_1) <_B (a_2)$.
\end{definition}

\begin{definition}
If $a_1 \leq_A a_2$, or if $a_1 - a_2$ and $b_1 <_B b_2$. It is called the \textbf{\uwave{dictionary order relation}} on $A \times B$. 
\end{definition}

With the order, we can do the comparison within the group. One example is the ``least upper bound property". One can define this property for an arbitrary ordered set. First, we need some preliminary definitions.

Suppose that $A$ is a set ordered by the relation $<$. Let $A_0$ be a subset of $A$. We say that the element $b$ is the \textbf{largest element} of $A_0$ if $b \in A_0$ and if $x \leq b$ for every  $x \in A_0$. Similarly, we say that is the \textbf{smallest element} of $A_0$ if $a \in A_0$. It is easy to see that \uwave{a set has at most one largest element and at most one smallest elements.}

We say that the subset $A_0$ of $A$ is \textbf{bounded above} if there is an element $b$ of $A$ such that $x \leq b$ for every $x \in A_0$; the element $b$ is called an \textbf{upper bound} of the subset $A_0$. If the set of all upper bound for $A_0$ has a smallest element, the element is called \textbf{the least upper bound}, or simply the \textbf{supremum}, of $A_0$. the supremum is denoted as $\sup{A_0}$;it may or may not belong to $A_0$. If it does, it is the largest element of $A_0$.

Similarly, we say that the subset of $A_0$ of $A$ is \textbf{bounded below} if there is an element $b$ of $A$ such that $x \geq b$ for every $x \in A_0$; the element $b$ is called an \text{lower bound} of the subset $A_0$. If the set of all lower bound for $A_0$ has a largest element, the element is called \textbf{the greatest lower bound}, or the \textbf{infimum}, of the subset $A_0$. It is denoted by $\inf{A_0}$; it may or may not belong to $A_0$. If it does, it is the smallest element of $A_0$.

\begin{definition}
An ordered set $A$ is said to have the \textbf{least upper bound property} if every nonempty subset $A_0$ of $A$ that is bounded above has a least upper bound. Analogously, the set $A$ is said to have the \textbf{\uwave{greatest lower bound property}} if every nonempty subset $A_0$ of $A$ that is bounded below has a greatest lower bound.
\end{definition}

\subsection{The Integers and the Real Numbers}\label{sec:the_integers_and_the_real_numbers}

Up to now we have been discussing what might be called the logical foundations for our study - the elementary concepts of set theory. Now we turn to what we might call the mathematical foundation for our study - the integers and the real number system.  Real number system contains multiple axioms, in this section, we will introduce these axioms and indicate how the familiar properties of real numbers and the integers are derived from them.

\begin{definition}[binary opration]
A \textbf{\uwave{binary operation}} on a set $A$ is a function $f$ mapping $A \times A$ into $A$.
\end{definition}

When dealing with a binary operation $f$ on a set $A$, we usually use a notation different from the standard functional notation introduce in section \ref{sec:functions}. Instead of denoting the value of the function $f$ at the point $(a, a^{\prime})$ by $f(a, a^{\prime})$, we usually write the symbol for the function between the two coordinates of the point in question, writing the value of the function at $(a, a^{\prime})$ as $afa^{\prime}$. Furthermore (just as was the case with relations), it is more common to use some symbol other than a letter to denote an operation. Symbols often used are the plus symbol $+$, the multiplication symbols $\cdot$ and $\circ$, and the asterisk $*$; however, there are many others.

\paragraph{Assumption} We assume there exists a set $\R$, called the set of \textbf{real numbers}, two binary operations $+$ and $\cdot$ on $\R$, called the addition and multiplication operations, respectively, and an order relation $<$ on $\R$, such that the following properties hold:

\begin{enumerate}[itemsep=0pt]
    \item Algebraic Properties
    \begin{enumerate}[itemsep=0pt]
        \item $(x + y) + z = x + (y + z)$
        \item $(x \cdot y) \cdot z = x \cdot (y \cdot z)$ for all $x, y, z \in \R$
    \end{enumerate}
    \begin{enumerate}[itemsep=0pt]
        \item $x + y = y + x$
        \item $x \cdot y = y \cdot x$ for all $x, y \in \R$
    \end{enumerate}
    \begin{enumerate}[itemsep=0pt]
        \item There exists a unique element of $\R$ called \textbf{zero}, denoted by $0$, such that $x + 0 = x$ for all $x \in \R$.
        \item There exists a unique element of $\R$ called \textbf{one}, different from $0$ and denoted by $1$, such that $x \cdot 1 = x$ for all $x \in \R$.
        \item For each $x$ in $\R$, there exists a unique $y$ in $\R$ such that $x + y = 0$.
        \item For each $x$ in $\R$ different from $0$, there exists a unique $y$ in $\R$ such that $x \cdot y = 1$.
        \item $x \cdot (y + z) = (x \cdot y) + (x \cdot z)$ for all $x, y, z\in \R$
    \end{enumerate}
    \item A Mixed Algebraic and Order Property
    \begin{enumerate}[itemsep=0pt]
        \item If $x > y$, then $x + z > y + z$
        \item If $x > y$ and $z > 0$, then $x \cdot z > y \cdot z$
    \end{enumerate}
    \item Order Properties 
    \begin{enumerate}[itemsep=0pt]
        \item The order relation $<$ has the least upper bound property
        \item If $x < y$, there exists an element $z$ such that $x < z$ and $z < y$.
    \end{enumerate}
\end{enumerate}

From the properties of $1$ follow the familiar ``laws of algebra". Given $x$, we can find:
\begin{enumerate}[itemsep=0pt]
    \item \textbf{negative of $x$}: denotes by $-x$ such that $x + (-x) = 0$
    \item \textbf{subtraction operation} by the formula $z - x = z + (-x)$
    \item \textbf{reciprocal} is denoted by $\frac{1}{x}$
    \item \textbf{quotient} is denoted by $\frac{z}{x} = z \cdot \frac{1}{x}$
\end{enumerate}

\begin{definition}[inductive and positive integers]
A subset $A$ of the real numbers is said to be \textbf{\uwave{inductive}} if it contains the number $1$, and if for every $x \in A$, the number $x + 1$ is also in $A$. Let $\mathcal{A}$ be the collection of all inductive subsets of $\R$. Then the set $\Z_+$ of \textbf{\uwave{positive integers}} is defined by the equation $\Z_+ = \bigcap_{A \in \mathcal{A}} A$.
\end{definition}

Notice that the set $\R_+$ of positive real numbers is inductive, for it contains $1$ and the statement $x > 0$ implies the statement $x + 1 > 0$. Therefore, $\Z_+ \subset \R_+$, so the elements of $\Z_+$ are indeed positive, as the choice of terminology suggests. The basic properties of $\Z_+$, which follow readily from the definition, are the following:
\begin{enumerate}[itemsep=0pt]
    \item $\Z_+$ is inductive
    \item (Principle of induction). If $A$ is an inductive set of positive integers, then $A = \Z_+$
\end{enumerate}

We define the set $\Z$ of \textbf{integers} to be the set consisting of the positive integers, $0$ and negative integers. We can easily prove that the sum, difference(subtraction), and product of two integers are integers, however, the quotient is not necessarily an integer. The set $\Q$ of quotient of integers is called the set of \textbf{rational numbers}.

\begin{theorem}[Well-ordering property]
Every nonempty subset of $\Z_+$ has a smallest element.
\end{theorem}

We will skip the proof.

\begin{theorem}[Strong induction principle]
Let $A$ be a set of positive integers. Suppose that for each positive integer $n$, the statement $S_n \subset A$\footnote{$S_n$ is a set of all integers less than $n$, formally, $S_n = \{1, 2, 3, \dots, n -1\}$} implies the state $n \in A$. Then $A = Z_+$
\end{theorem}

We will skip the proof.

\subsection{Cartesian Products}

We have already define the cartesian product with two sets: $A \times B$, now we introduce more general cartesian products.

\begin{definition}[indexing function and index set]
Let $\mathcal{A}$ be a nonempty collection of sets. An \textbf{indexing function} for $\mathcal{A}$ is a surjective function $f$ from some set $J$, called the \textbf{index set}, to $\mathcal{A}$. The collection $\mathcal{A}$, together with the index function $f$, is called an \textbf{indexed family of sets}. Given $\alpha \in J$, we shall denote the set $f(\alpha)$ to be the symbol $A_{\alpha}$, and we shall denote the indexed family itself by the symbol $\{A_\alpha\}_{\alpha \in J}$, which is read ``the family of all $A_{\alpha}$, as $\alpha$ ranges over $J$." Sometimes we can directly denote the indexed family as $\{A_{\alpha}\}$.
\end{definition}

Note that although an indexing function is required to be \uwave{surjective}, it is not required to be injective. It is entirely possible for $A_{\alpha}$ and $A_{\beta}$ to be the same set of $\mathcal{A}$, even though $\alpha \neq \beta$.

One way in which indexing function are used is to give a new notation for arbitrary unions and intersections of sets. Suppose that $f: J \mapsto \mathcal{A}$ is an indexing function for $\mathcal{A}$; let $A_{\alpha}$ denote $f(\alpha)$. Then we define $\bigcup_{\alpha \in J} = \{x: \text{ for at least one } \alpha \in J, x \in A_{\alpha}\}$ and $\bigcap_{\alpha \in J} = \{x: \text{ for every } \alpha \in J, x \in A_{\alpha}\}$.

Two especially useful index sets are the set $\{1, \dots, n\}$ and the set $\Z_+$. For these index sets, we introduce some special notation. If a collection of sets is index by the $\{1, \dots, n\}$, we denote the index family by the symbol $\{A_1, \dots, A_n\}$, and we denote the union and intersection as $A_1 \cup \dots \cup A_n$ and $A_1 \cap \dots \cap A_n$, respectively. For the usage of $\Z_+$, the index family will be denoted by the symbole $\{A_1, \dots, A_n, \dots\}$ and its union and intersection as $A_1 \cup \dots \cup A_n, \cup \dots$ and $A_1 \cap \dots \cap A_n \cap \dots$, respectively.

\begin{definition}[$m$-tuple]
Let $m$ be a positive integer. Given a set $X$, we define an \textbf{\uwave{$m$-tuple}} of elements of $X$ to be a function: $x: \{1, \dots, m\} \mapsto X$.
\end{definition}

If $x$ is an $m$-tuple, we often denote the value of $x$ at $i$ by the symbol $x_i$ rather than $x(i)$ and call it the $i$-th coordinate or $x$. And we often denote the function $x$ itself by the symbol $(x_1, \dots, x_m)$.

Now let $\{A_1, \dots, A_m\}$ be a family of sets indexed with the set $\{1, \dots, m\}$. Let $X = A_1 \cup A_2 \cup \dots A_m$, and define the \textbf{\uwave{cartesian product}} of the indexed family, denoted by $\prod_{i = 1}^{m} A_i$ to be the set of all $m$-tuples $(x_1, \dots, x_m)$ of element of $X$, such that $x_i \in A_i$ for each $i$.

\begin{definition}[$\omega$-tuple]
Given a set $X$, we define an $\omega$-tuple of elements of $X$ to be a function $x: \Z_{+} \maps to X$; we also call such a function a \textbf{sequence}, or an \textbf{infinite sequence}, of element of $X$. If $x$ is an $\omega$-tuple, we often denote the value of $x$ at $i$ by $x_i$, and call it the $i$-th coordinate of $x$. 
\end{definition}

Now with the $\omega$-tuple, the \textbf{cartesian product} of the indexed family of sets, denoted by $\prod_{i \in \Z_+} A_i$ is defined to be the set of all $\omega$-tuples $(x_1, x_2, \dots)$ of elements of $X$ such that $x_i \in A_i$ for each $i$.

\subsection{Finite Sets}

Finite sets and infinite sets, countable sets and uncountable sets, these are types of sets that you may have encountered before. Nevertheless, we shall discuss them in this section and the next, not only to make sure you understand them thorughly, but also to elucidate some particular points of logic that will arise later on. First, we consider the finite sets.

Re call that if $n$ is a positive integer, we use $S_n$ to denote the set of positive integers less than $n$; it is call a \textbf{section} of the positive integers. The sets $S_n$ are the prototypes for what we call the finite sets.

\begin{definition}[finite set]
A set is said to be \textbf{\uwave{finite}} if there is a bijective correspondence of $A$ with some section of the positive integers. That is, $A$ is finite if it is empty of if there is a bijection $f: A \mapsto \{1, \dots, n\}$
\end{definition}

To clarify the definition, if a set is \textbf{finite}, it has a finite number of elements. For a set $A$ contains $n$ elements, its \textbf{\uwave{cardinality}} is $n$.

\begin{lemmad}
Let $n$ be a positive integer. Let $A$ be a set; let $a_0$ be an element of $A$. Then there exists a bijective correspondence $f$ of the set $A$ with the set $\{1, \dots, n+1\}$ if and only if there exists a bijective correspondence $g$ of the set $A \backslash \{a_0\}$ with the set $\{1, \dots, n\}$.
\end{lemmad}

\begin{theorem}
Let $A$ be a set; suppose there exists a bijection $f: A \mapsto \{1, \dots, n\}$ for some $n \in \Z_+$. Let $B$ be a proper subset of $A$. Then there exists no bijection $g: B \mapsto \{1, \dots, n\}$; however, there is a bijection $g: B \mapsto m$ where $m < n$ if $B \neq \varnothing$.
\end{theorem}

\begin{corollary}
If $A$ is finite, there is no bijection of $A$ with a proper subset of itself
\end{corollary}

\begin{corollary}
$\Z_+$ is not finite.
\end{corollary}

\begin{corollary}
The cardinality of a finite set $A$ is uniquely determined by $A$.
\end{corollary}

\begin{corollary}
If $B$ is a subset of the finite set $A$, then $B$ is finite. If $B$ is a proper subset of $A$, then the cardinality of $B$ is less than the cardinality of $A$.
\end{corollary}

\begin{corollary}
Let $B$ be a nonempty set. Then the following are equivalent:
\begin{enumerate}[itemsep=0pt]
    \item $B$ is finite
    \item There is a surjective function from a section of the positive integers onto $B$
    \item There is an injective function from $B$ into a section of the positive integers
\end{enumerate}
\end{corollary}

\begin{corollary}
Finite unions and finite cartesian products of finite sets are finite.
\end{corollary}

\subsection{Countable and Uncountable Sets}

Just as sections of the positive integers are the prototypes for the finite sets, the set of all the positive integers is the prototype for what we call the \textbf{countably infinite} sets. In this section, we shall study such set; we shall also construct some sets that are neither finite nor countably infinite. This study will lead us into a discussion of what we mean by the process of ``inductive definition".

\begin{definition}[infinite and countably infinite]
A set $A$ is said to be \textbf{\uwave{infinite}} if it is not finite. It is said to be \textbf{\uwave{countably infinite}} if there is a bijective correspondence $f:A \mapsto \Z_{+}$.
\end{definition}

\begin{definition}[countable and uncountable]
A set is said to be \textbf{\uwave{countable}} if it is either finite or countable infinite. A set that is not countable is said to be \textbf{\uwave{uncountable}}.
\end{definition}

There is a very useful criterion for showing that a set is countable. It is the following:

\begin{theorem}
Let $B$ be a nonempty set. Then the following are equivalent:
\begin{enumerate}[itemsep=0pt]
    \item $B$ is countable
    \item There a surjective function $f:\Z_+ \mapsto B$
    \item There is an injective function $g: B \mapsto \Z_+$
\end{enumerate}
\end{theorem}

\begin{lemmad}
If $C$ is an infinite subset of $\Z_+$, then $C$ is countably infinite.
\end{lemmad}

\begin{corollary}
A subset of a countable set is countable
\end{corollary}

\begin{corollary}
The set $\Z_+ \times \Z_+$ is countably infinite
\end{corollary}

\begin{theorem}
A countable union of countable sets is countable
\end{theorem}

\begin{theorem}
A finite product of countable sets is countable
\end{theorem}

\begin{theorem}
Let $X$ denote the two element set $\{0, 1\}$. Then the set $X^{\omega}$ is uncountable.
\end{theorem}

\begin{theorem}
Let $A$ be a set, there is no injective map $f: \mathcal{P}(A) \mapsto A$ and there is no surjective map $g: A \mapsto \mathcal{P}(A)$
\end{theorem}



\section{Topological Spaces and Continuous Functions}

\section{Connectedness and Compactness}
